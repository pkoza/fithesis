%%%%%%%%%%%%%%%%%%%%%%%%%%%%%%%%%%%%%%%%%%%%%%%%%%%%%%%%%%%%%%%%%%%%
%% I, the copyright holder of this work, release this work into the
%% public domain. This applies worldwide. In some countries this may
%% not be legally possible; if so: I grant anyone the right to use
%% this work for any purpose, without any conditions, unless such
%% conditions are required by law.
%%%%%%%%%%%%%%%%%%%%%%%%%%%%%%%%%%%%%%%%%%%%%%%%%%%%%%%%%%%%%%%%%%%%

\documentclass[
  digital, %% This option enables the default options for the
           %% digital version of a document. Replace with `printed`
           %% to enable the default options for the printed version
           %% of a document.
  table,   %% Causes the coloring of tables. Replace with `notable`
           %% to restore plain tables.
  lof,     %% Prints the List of Figures. Replace with `nolof` to
           %% hide the List of Figures.
  lot,     %% Prints the List of Tables. Replace with `nolot` to
           %% hide the List of Tables.
  %% More options are listed in the user guide at
  %% <http://mirrors.ctan.org/macros/latex/contrib/fithesis/guide/mu/fi.pdf>.
]{fithesis3}
%% The following section sets up the locales used in the thesis.
\usepackage[resetfonts]{cmap} %% We need to load the T2A font encoding
\usepackage[T1,T2A]{fontenc}  %% to use the Cyrillic fonts with Russian texts.
\usepackage[
  main=slovak, %% By using `czech` or `slovak` as the main locale
                %% instead of `english`, you can typeset the thesis
                %% in either Czech or Slovak, respectively.
  german, russian, czech, english %% The additional keys allow
]{babel}        %% foreign texts to be typeset as follows:
%%
%%   \begin{otherlanguage}{german}  ... \end{otherlanguage}
%%   \begin{otherlanguage}{russian} ... \end{otherlanguage}
%%   \begin{otherlanguage}{czech}   ... \end{otherlanguage}
%%   \begin{otherlanguage}{slovak}  ... \end{otherlanguage}
%%
%% For non-Latin scripts, it may be necessary to load additional
%% fonts:
\usepackage{paratype}
\usepackage{bera}% optional: just to have a nice mono-spaced font
\usepackage{listings}
\usepackage{xcolor}

\colorlet{punct}{red!60!black}
\definecolor{background}{HTML}{EEEEEE}
\definecolor{delim}{RGB}{20,105,176}
\colorlet{numb}{magenta!60!black}

\lstdefinelanguage{json}{
	basicstyle=\normalfont\ttfamily,
	numbers=left,
	numberstyle=\scriptsize,
	stepnumber=1,
	numbersep=8pt,
	showstringspaces=false,
	breaklines=true,
	frame=lines,
	backgroundcolor=\color{background},
	literate=
	*{0}{{{\color{numb}0}}}{1}
	{1}{{{\color{numb}1}}}{1}
	{2}{{{\color{numb}2}}}{1}
	{3}{{{\color{numb}3}}}{1}
	{4}{{{\color{numb}4}}}{1}
	{5}{{{\color{numb}5}}}{1}
	{6}{{{\color{numb}6}}}{1}
	{7}{{{\color{numb}7}}}{1}
	{8}{{{\color{numb}8}}}{1}
	{9}{{{\color{numb}9}}}{1}
	{:}{{{\color{punct}{:}}}}{1}
	{,}{{{\color{punct}{,}}}}{1}
	{\{}{{{\color{delim}{\{}}}}{1}
	{\}}{{{\color{delim}{\}}}}}{1}
	{[}{{{\color{delim}{[}}}}{1}
	{]}{{{\color{delim}{]}}}}{1},
}

\def\textrussian#1{{\usefont{T2A}{PTSerif-TLF}{m}{rm}#1}}
%%
%% The following section sets up the metadata of the thesis.
\thesissetup{
    date          = \the\year/\the\month/\the\day,
    university    = mu,
    faculty       = fi,
    type          = bc,
    author        = Peter Koza,
    gender        = m,
    advisor       = RNDr. Jaroslav Ráček{,} Ph.D.,
    title         = {Portál digitálního kulturního dědictví},
    TeXtitle      = {Portál digitálního kulturního dědictví},
    keywords      = {keyword1, keyword2, ...},
    TeXkeywords   = {keyword1, keyword2, \ldots},
}
\thesislong{abstract}{
    Popsat principy digitalizace a publikování oběktů kulturního dědictví. Zaměřit se na tuto problematiku zejména z pohledu regionálních institucí v České republice. Popsat a analyzovat možnosti využití technologie Liferay Portal pro zpracování a publikování záznamů o objektech kulturního dědictví. V praktické části analyzovat, navrhnout, implementovat a nasadit do provozu prototyp konkrétního portálu kulturního dědictví. Analýza a návrh systému bude nad modely UML, implementace bude v Liferay Portál. V rámci práce řešit i vyhledávání nad automaticky rozpoznanými texty jednotlivých objektů a import dat z prostředí vybrané instituce.
}
\thesislong{thanks}{
    This is the acknowledgement for my thesis, 
    
    which can never span multiple paragraphs.
}
%% The following section sets up the bibliography.
\usepackage{csquotes}
\usepackage[              %% When typesetting the bibliography, the
  backend=biber,          %% `numeric` style will be used for the
  style=numeric,          %% entries and the `numeric-comp` style
  citestyle=numeric-comp, %% for the references to the entries. The
  sorting=none,           %% entries will be sorted in cite order.
  sortlocale=auto         %% For more unformation about the available
]{biblatex}               %% `style`s and `citestyles`, see:
%% <http://mirrors.ctan.org/macros/latex/contrib/biblatex/doc/biblatex.pdf>.
\addbibresource{example.bib} %% The bibliograpic database within
                          %% the file `example.bib` will be used.
\usepackage{makeidx}      %% The `makeidx` package contains
\makeindex                %% helper commands for index typesetting.
%% These additional packages are used within the document:
\usepackage{paralist}
\usepackage{amsmath}
\usepackage{amsthm}
\usepackage{amsfonts}
\usepackage{url}
\usepackage{menukeys}
\begin{document}
\chapter{Portály digitálneho kultúrneho dedičstva}	
Zámerom tejto iniciatívy je vytvoriť webový portál, ktorého prostredníctvom bude zaisťovaná príprava a samotné sprístupnenie digitálneho obsahu vybraných fondov pamäťových inštitúcií pôsobiacich na území Českej republiky širokej verejnosti a odborným bádateľom. Ďalej bude umožňovať on-line úpravu metadát a možnosť priloženia ďalších materiálov ktoré s dokumentom súvisia. Web by mal užívateľa zaujať a vytvoriť dojem jednoducho používateľnej interaktívnej stránky aj u používateľa, ktorého daná tématika nezaujíma.
\section{Problémy existujúcich riešení}
Digitálny obsah sa primárne skladá z kníh. Problémom existujúcich nástrojov na prehliadanie tohto obsahu je nepríťažlivosť pre užívateľa. 
\subsection{Zobrazenie knižných dát}
Tabuľkové zobrazenie knižných záznamov nespĺňa požiadavky modernej webovej aplikácie. Príkladom je katalóg Slezského zemského múzea\footnote{Viď \url{http://knihovna.szmo.cz/katalog/}.}. Vyhľadávanie je realizované jedným formulárom. Výsledky sú zobrazované jednoduchou tabuľkou, ktorá poskytuje minimum informácií o dokumente a jeho uložení. V tabuľke sú ako prvé uvedené informácie Dok a Sign, ktorých názvy sú neintuitívne a teda pre koncového užívateľa pri prvom prístupe na web nepodstatné. Ďalej sú zobrazené údaje autor, názov, časť, rok a počet. Zvyšok údajov je zobrazený až po kliknutí na názov dokumentu. Ostatné atribúty sú neaktívne. V detaile dokumentu chýbajú užívateľsky príťažlivé prvky, ako napríklad mapa uloženia, diskusia k dielu, možnosť rezervácie alebo zobrazenie skenovaných obrázkov. Aj napriek tomu, že tento web podáva hodnoverné informácie o dokumentoch, nie je možné ho označiť za užívateľsky príťažlivý a po prvom použití nenavádza užívateľa k ďalšej návšteve.
\subsection{Realizácia vyhľadávania}
Na stránke systému Kramerius\footnote{Viď \url{http://kramerius.nkp.cz/kramerius/Welcome.do}.} je sprístupnené vyhľadávanie dokumentov Národnej knižnice Českej republiky. Užívateľské rozhranie je oproti predošlému webu obohatené o funkciu hľadania v celom texte, ktorá umožňuje jednoduchšie zoznámenie sa s funkciami aplikácie. Pokročilé vyhľadávanie je možné zobraziť a skryť pridaným tlačidlom. Priamo na úvodnej stránke sú umiestnené príklady skenovaných dokumentov, ktoré odkazom vedú na detail diela. Zobrazenie výsledkov vyhľadávania je riešené jednoduchým zoznamom. Tento prístup prináša vysokú mieru neprehľadnosti, pretože užívateľ nemôže porovnať hodnoty, či zoradiť výsledky na základe jednotlivých atribútov. Detail obsahuje
stručné informácie o diele a náhľad strán, ktorý ale nie je dostupný bez inštalácie zásuvného modulu. 
\section{Poučenie o tvorbe portálov kultúrneho dedičstva}
Z analýzy existujúcich nástrojov na prehliadanie kultúrneho dedičstva bolo zistené že pri návrhu aplikácie by sme sa mali zamerať na nasledujúce oblasti:
\begin{itemize}
	\item vyhľadávanie
	\begin{itemize}
		\item umožniť hľadanie v celom texte 
		\item pridať prvky pre oživenie ako napr. mapa alebo rezy
	\end{itemize}
	\item zobrazenie výsledkov
	\begin{itemize}
		\item vytvoriť prehľadnú tabuľku obsahujúcu relevantné atribúty
		\item v prípade použitia mapy poskytnúť možnosť jednoduchého prepnutia 
	\end{itemize}
	\item detail diela
	\begin{itemize}
		\item pridať užívateľsky príťažlivé prvky
		\begin{itemize}
			\item mapa umiestnenia				
			\item diskusia k dielu
			\item zobrazenie náhľadu dokumentov
			\item rezervácia knihy
		\end{itemize}
	\end{itemize}	
\end{itemize}

\chapter{Analýza}
\section{Požiadavky zákazníka}
Po komunikácii so zákazíkom na AWS boli zistené požiadavky zákazníka zobrazené diagramom na obrázku 546846

Use case diagram jak hovado

\subsection{Kategórie dokumentov}
Dokumenty , ktoré sú predmetom prípravy a zverejnenia prostredníctvom portálu digitálneho kultúrneho dedičstva(ďalej už len PD) sú rozdelené do dvoch základných kategórií:
\begin{itemize}
	\item dokumenty archívnych fondov a múzejných zbierok				
	\item dokumenty knižničných fondov podľa štandardov NDK\footnote{Národní digitální knihovna}
\end{itemize}
Pre vyššie uvedené jednotlivé kategórie je nutné použiť v PD rozdielne pracovné postupy a to hlavne v prípravnej fáze pred zverejnením. Dôvodom je samotná podstata dokumentu z pohľadu toho, či ide o dokumenty, ktoré sú v nemennej podobe (monografie, periodiká) alebo dokumenty, u ktorých sa v čase môžu meniť predovšetkým ich popisné informácie. Pre každú kategóriu sú taktiež definované odlišné dátové štruktúry metadát. Predpokladaný objem dokumentov spravovaných PD je 100 000 jednotlivých dokumentov, 1 500 000 strán prepočítaných na formát A4 a približne 20 terabajtov obrazových dát.
\subsection{Dokumenty archívnych fondov a múzejných zbierok}
Druhy dokumentov:
\begin{itemize}
	\item fotografie, negatívy				
	\item zväzky - kroniky, zápisy, katalógy
	\item voľné archy - dokumenty, plagáty, plány
	\item staré výtlačky, spevníky
	\item kartografický materiál 
	\item filmy, zvukové nahrávky 
	\item zbierkové predmety	
\end{itemize}
Jedná sa o dokumenty, ktoré vznikli digitalizáciou vybraných archívnych fondov a múzejných zbierok. Z hľadiska nemennosti týchto dokumentov sa jedná o dokumenty premenlivej povahy. Popisné metadáta z KDJ\footnote{Krajská digitalizační jednotka} pre PD u tejto kategórie preto nie je možné použiť. Jedná se teda o dokumenty, u ktorých je nutné ručné doplnenie metadát. 

\subsection{Dokumenty knižničných fondov podľa štandardov NDK}
Druhy dokumentov:
\begin{itemize}
	\item monografie - jednodielne alebo viacdielne knižné dokumenty			
	\item periodiká - pravidelne vychádzajúce výtlačky
\end{itemize}
Jedná sa o dokumenty prevažne nemennej povahy. Predpokladá se teda, že ako uložené digitálne obrazy, tak metadáta majú konečný a nemenný stav. Všetky tieto dokumenty sú uložené v úložisku KDJ vo formáte SIP. KDJ bude pre tento typ dokumentov zdrojovým poskytovateľom informácií pre PD. 
\subsection{Typy dokumentov} 
Z hľadiska typu súborov, v ktorých sú digitálne dokumenty uložené, se jedná o dokumenty:
\begin{itemize}
	\item obrazové			
	\item videá
	\item zvukové záznamy
	\item vo formáte XML(popisné metadáta)		
	\item vo formáte PDF
\end{itemize}
\subsection{Vstup dokumentov} 
Táto časť PD bude zabezpečovať riadený import dát pre prípravu sprístupnenia dokumentov. Vstupné mechanizmy budú rešpektovať potreby jednotlivých kategórií dokumentu. Po importe budú dáta pripravených dokumentov uložené do zverejňovacej databázy. Princíp predávania dokumentov, detailný popis rozhrania a jeho dátová štruktúra bude súčasťou implementačnej analýzy. Zákazník špecifikoval nasledujúce podmienky:
\begin{itemize}
	\item PD bude umožňovať import viacerých dokumentov súčasne.
	\item Pri zahájení prípravy vstupu dokumentu do PD bude prevedená kontrola duplicity s predchádzajúcimi importovanými dokumentmi.
	\item Behom importu dokumentu do PD systém priradí jeho vlastníka – osobu kompletne zodpovednú za celé sprístupnenie a zverejnenie. Vlastníkom se stáva užívateľ PD, ktorý import úspešne dokončil. Zdrojom dát pre import bude mimo iných webové rozhranie PD a zdieľané diskové zložky unikátne pre každého administrátora.
	\item PD bude disponovať funkciou pre import a aktualizáciu popisných metadát a prípadných obrazových dát zo systému JANUS2000 (evidenčný systém okresných archívov) a systému VISMO (dáta projektu MG Vysočiny on-line).
\end{itemize}
\subsection{Príprava dokumentov k sprístupneniu}
Táto časť PD bude zahŕňať proces sprístupnenia dokumentov a možnosť ich úpravy pred zverejním. Zákazník špecifikoval nasledujúce funkcie:
\begin{itemize}
	\item PD po prihlásení do časti pre prípravu dokumentu ponúkne oprávnenému užívateľovi zoznam všetkých jeho dokumentov s informáciou o ich stave a možnosťou zverejnenia dokumentu pre všetkých užívateľov.
	\item PD umožní oprávnenému užívateľovi zobraziť náhľad upraveného dokumentu, znázorňujúci stav, v ktorom bude dokument zverejnený.
	\item PD umožňuje oprávnenému užívateľovi kedykoľvek zrušiť zverejnenie dokumentu. 
	\item Oprávnený užívateľ, vlastník dokumentu, má na karte otvoreného dokumentu k dispozícii históriu úprav.
	\item Portál PD umožní oprávnenému užívateľovi možnosť úpravy popisných dát podľa úrovne jeho oprávnení.
	\item Štruktúra dokumentu bude navrhnutá podľa štandardu Dublin Core.
	\item Všetky dôležité akcie PD ako sprístupnenie, úprava alebo zmazanie dokumentu budú sprevádzané notifikačným e-mailom pre všetkých dotknutých užívateľov.
\end{itemize}
\subsection{Operácie nad zverejnenými dokumentmi}
Táto časť portálu obsahuje funkcie dostupné pre širokú verejnosť, tzn. registrovaných a neregistrovaných užívateľov. Zákazník špecifikoval nasledujúce funkcie, dostupné všetkým užívateľom:
\begin{itemize}
	\item PD bude poskytovať možnosť vyhľadávania v celom texte.
	\item PD bude poskytovať možnosť vyhľadávania podľa konkrétnych atribútov.
	\item PD umožní export obsahu príslušného dokumentu do PDF súboru.
\end{itemize}
Nasledujúce funkcie budú dostupné všetkým registrovaným užívateľom:
\begin{itemize}
	\item Možnosti pre užívateľský popis príslušného dokumentu. Ide o diskusiu k dielu a prepis nerozpoznaného textu dokumentu.
	\item PD umožní rezerváciu diela pre vypožičanie.
\end{itemize}
\section{Návrh riešenia}
Na základe analýzy existujúcich nástrojov na prehliadanie digitálnych fondov a komunikácie so zákazníkom bol vytvorený konečný návrh projektu. Pre implementáciu PD bude využité open source
portálové riešenie \textit{Liferay} vo verzii \textit{6.2.2 community edition GA3}. Prípadný prechod na vyššiu verziu je možný prostredníctvom štandardných postupov pre portál Liferay a nieje zahrnutý v rozsahu tohto projektu. Zaznamenávanie operácií v PD bude zabezpečené pomocou frameworku pre Java aplikácie \textit{Log4j}. Výstupné hlášky sú štandardne zaznamenávané do súboru na aplikačnom serveri, na ktorom aplikácia beží. Jednotlivé hlášky budú rozdelené do úrovní. Nastavovanie týchto úrovní bude možné prostredníctvom administračného rozhrania portálu Liferay. Administrácia zverejňovania digitálneho obsahu bude dostupná všetkým oprávneným užívateľom. Vyhľadávanie bude dostupné všetkým užívateľom bez ohľadu na ich identitu.

\subsection{Grafický vzhľad}
Grafický vzhľad aplikácie bude meniteľný pomocou špeciálnych zásuvných modulov podporovaných portálom Liferay (tzv. témy). Zmena vzhľadu vyžaduje vývoj alebo úpravu zásuvného modulu.
\chapter{Technológie}
\section{Indexácia}
\subsection{Výber vyhľadávacieho nástroja}
Pre potreby indexácie boli v analytickej fáze zvažované tieto možnosti:
\begin{itemize}
	\item \textit{Elastic} – vyhľadávací nástroj postavený na Java knižnici \textit{Apache Lucene}
	\item \textit{Sphinx} – vyhľadávací nástroj vytvorený v jazyku C++
\end{itemize}
Po konzultácii s vývojovým tímom sa nakoniec rozhodlo pre \textit{Elastic} z dôvodu vyššej spoľahlivosti a jednoduchého používania.
\subsection{Elastic}
Elastic je nástroj na indexáciu dát. Jeho primárnym účelom je vyhľadávanie nad relatívne veľkým množstvom textu. Oproti databáze ponúka oveľa rýchlejšiu odpoveď a rozvinutejšie možnosti štrukturovaného ukladania dát. 

Aplikácie využívajúce Elastic s ním komunikujú pomocou HTTP protokolu a dotazy ako  aj odpovede sú vo formáte JSON. Príklad dotazu na Elastic:


\begin{lstlisting}[language=json,firstnumber=1]
GET
{
"query" : {
	"term" : { "gender": "female" },
	"range" : {
		"age" : {
		"from" : 20,
		"to" : 30
		}
	} 
}}
	\end{lstlisting}
Tento dotaz vyhľadá všetky záznamy s uvedenými parametrami. \textit{Term} je označenie výrazu, ktorý vyžaduje presný obsah určitého atribútu. \textit{Range} vyhľadá všetky záznamy ktoré sa nachádzajú v určenom rozmedzí. V tomto prípade hľadáme ženu od 20 do 30 rokov. Odpoveďou na takýto dotaz by mohlo byť napríklad:
\begin{lstlisting}[language=json,firstnumber=1]
"hits":{
  "total" : 1, 
	"hits" : [
		  {"_index" : "pdkd",
		   "_type" : "person",
		   "_id" : "1",
		   "_source" : { 
			  "name" : "Kim", 
			  "age" : 22,
			  "gender" : "female" 
		  }
	} 
	]
}
\end{lstlisting}
Výsledkom dotazu je teda 1 záznam. 
\subsection{Ďalšie možnosti vyhľadávania}							
\paragraph{Radenie}
Poradie vrátených záznamov môže byť vytvorené na základe určitého atribútu, relevancie výsledku alebo kombinácie týchto faktorov.
\paragraph{Score}
Ďalšou zaujímavou funkciou elasticu je \textit{relevance scoring}. Po tom čo získame zoznam vyhovujúcich záznamov, je potrebné ich zoradiť podľa relevancie. V prípade kombinovaného dotazu s logickými spojkami neobsahuje každý záznam všetky podmienky. Skóre relevancie sa počíta na základe troch faktorov:
\begin{compactenum}
	\item \textbf{Frekvencia výskytu hľadaného výrazu v zázname} - čím viac je v zázname zmienená, tým je záznam podstatnejší.
	\item \textbf{Inverzná frekvencia hľadaného výrazu} – relevancia výrazu sa znižuje v závislosti od množstva záznamov obsahujúcich tento výraz.
	\item \textbf{Rozsah atribútu , v ktorom bol výraz nájdený} – čím je obsah atribútu dlhší, tým je relevancia tohto výskytu nižšia
\end{compactenum}
\paragraph{Aggregations}			
Výsledky vyhľadávania je niekedy potrebné zlučovať do skupín na základe určitého parametru. \textit{Elastic} ponúka funkciu \textit{Aggregations}. Pokiaľ je teda v dotaze špecifikovaný atribút, na základe ktorého sa majú záznamy zoskupovať je výsledkom objekt obsahujúci niekoľko skupín ďalších objektov, ktoré môžu predstavovať samotné záznamy alebo ďalšie skupiny zjednotení. Vďaka formátu JSON je možné tieto objekty serializovať. V prípade objektových jazykov ako Java je potom jednoduché poskladať štruktúry, s ktorými jazyk pracuje pomocou knižnice na deserializáciu.  


\chapter{These are}
\section{the available}
\subsection{sectioning}
\subsubsection{commands.}
\paragraph{Paragraphs and}
\subparagraph{subparagraphs are available as well.}
Inside the text, you can also use unnumbered lists,
\begin{itemize}
  \item such as
  \item this one
  \begin{itemize}
    \item     and they can be nested as well.
    \item[>>] You can even turn the bullets into something fancier,
    \item[\S] if you so desire.
  \end{itemize}
\end{itemize}
Numbered lists are
\begin{enumerate}
  \item very
  \begin{enumerate}
    \item similar
  \end{enumerate}
\end{enumerate}
and so are description lists:
\begin{description}
  \item[Description list]
    A list of terms with a description of each term
\end{description}
The spacing of these lists is geared towards paragraphs of text.
For lists of words and phrases, the \textsf{paralist} package
offers commands
\begin{compactitem}
  \item that
  \begin{compactitem}
    \item are
    \begin{compactitem}
      \item better
      \begin{compactitem}
        \item suited
      \end{compactitem}
    \end{compactitem}
  \end{compactitem}
\end{compactitem}
\begin{compactenum}
  \item to
  \begin{compactenum}
    \item this
    \begin{compactenum}
      \item kind of
      \begin{compactenum}
        \item content.
      \end{compactenum}
    \end{compactenum}
  \end{compactenum}
\end{compactenum}
The \textsf{amsthm} package provides the commands necessary for the
typesetting of mathematical definitions, theorems, lemmas and
proofs.

%% We will define several mathematical sectioning commands.
\newtheorem{theorem}{Theorem}[section] %% The numbering of theorems
                               %% will be reset after each section.
\newtheorem{lemma}[theorem]{Lemma}     %% The numbering of lemmas
\newtheorem{corr}[theorem]{Corrolary}  %% and corrolaries will
                                %% share the counter with theorems.
\theoremstyle{definition}
\newtheorem{definition}{Definition}
\theoremstyle{remark}
\newtheorem*{remark}{Remark}

\begin{theorem}
  This is a theorem that offers a profound insight into the
  mathematical sectioning commands.
\end{theorem}
\begin{theorem}[Another theorem]
  This is another theorem. Unlike the first one, this theorem has
  been endowed with a name.
\end{theorem}
\begin{lemma}
  Let us suppose that $x^2+y^2=z^2$. Then
  \begin{equation}
    \biggl\langle u\biggm|\sum_{i=1}^nF(e_i,v)e_i\biggr\rangle
    =F\biggl(\sum_{i=1}^n\langle e_i|u\rangle e_i,v\biggr).
  \end{equation}
\end{lemma}
\begin{proof}
  $\nabla^2 f(x,y)=\frac{\partial^2f}{\partial x^2}+
   \frac{\partial^2f}{\partial y^2}$.
\end{proof}
\begin{corr}
  This is a corrolary.
\end{corr}
\begin{remark}
  This is a remark.
\end{remark}

\chapter{Floats and references}
\begin{figure}
  \begin{center}
    %% PNG and JPG images can be inserted into the document as well,
    %% but their resolution needs to be adequate. The minimum is
    %% about 250 pixels per 1 centimeter. That means that a JPG or
    %% PNG image typeset at 40 × 40 mm should be 1000 × 1000 px
    %% large at minimum.
    \includegraphics[width=40mm]{fithesis/logo/mu/fithesis-base.pdf}
  \end{center}
  \caption{The logo of the Masaryk University at 40\,mm}
  \label{fig:mulogo1}
\end{figure}

\begin{figure}
  \begin{minipage}{.66\textwidth}
    \includegraphics[width=\textwidth]{fithesis/logo/mu/fithesis-base.pdf}
  \end{minipage}
  \begin{minipage}{.33\textwidth}
    \includegraphics[width=\textwidth]{fithesis/logo/mu/fithesis-base.pdf} \\
    \includegraphics[width=\textwidth]{fithesis/logo/mu/fithesis-base.pdf}
  \end{minipage}
  \caption{The logo of the Masaryk University at $\frac23$ and
    $\frac13$ of text width}
  \label{fig:mulogo2}
\end{figure}

\begin{table}
  \begin{tabularx}{\textwidth}{lllX}
    \toprule
    Day & Min Temp & Max Temp & Summary \\
    \midrule
    Monday & $13^{\circ}\mathrm{C}$ & $21^\circ\mathrm{C}$ & A
    clear day with low wind and no adverse current advisories. \\
    Tuesday & $11^{\circ}\mathrm{C}$ & $17^\circ\mathrm{C}$ & A
    trough of low pressure will come from the northwest. \\
    Wednesday & $10^{\circ}\mathrm{C}$ &
    $21^\circ\mathrm{C}$ & Rain will spread to all parts during the
    morning. \\
    \bottomrule
  \end{tabularx}
  \caption{A weather forecast}
  \label{tab:weather}
\end{table}

The logo of the Masaryk University is shown in Figure
\ref{fig:mulogo1} and Figure \ref{fig:mulogo2} at pages
\pageref{fig:mulogo1} and \pageref{fig:mulogo2}. The weather
forecast is shown in Table \ref{tab:weather} at page
\pageref{tab:weather}. The following chapter is Chapter
\ref{chap:matheq} and starts at page \pageref{chap:matheq}.
Items \ref{item:star1}, \ref{item:star2}, and
\ref{item:star3} are starred in the following list:
\begin{compactenum}
  \item some text
  \item some other text
  \item $\star$ \label{item:star1}
  \begin{compactenum}
    \item some text
    \item $\star$ \label{item:star2}
    \item some other text
    \begin{compactenum}
      \item some text
      \item some other text
      \item yet another piece of text
      \item $\star$ \label{item:star3}
    \end{compactenum}
    \item yet another piece of text
  \end{compactenum}
  \item yet another piece of text
\end{compactenum}
If your reference points to a place that has not yet been typeset,
the \verb"\ref" command will expand to \textbf{??} during the first
run of
\texttt{pdflatex \jobname.tex}
and a second run is going to be needed for the references to
resolve. With online services -- such as Overleaf -- this is
performed automatically.

\chapter{Mathematical equations}
\label{chap:matheq}
\TeX{} comes pre-packed with the ability to typeset inline
equations, such as $\mathrm{e}^{ix}=\cos x+i\sin x$, and display
equations, such as \[
  \mathbf{A}^{-1} = \begin{bmatrix}
  a & b \\ c & d \\
  \end{bmatrix}^{-1} =
  \frac{1}{\det(\mathbf{A})} \begin{bmatrix}
  \,\,\,d & \!\!-b \\ -c & \,a \\
  \end{bmatrix} =
  \frac{1}{ad - bc} \begin{bmatrix}
  \,\,\,d & \!\!-b \\ -c & \,a \\
  \end{bmatrix}.
\] \LaTeX{} defines the automatically numbered \texttt{equation}
environment:
\begin{equation}
  \gamma Px = PAx = PAP^{-1}Px.
\end{equation}
The package \textsf{amsmath} provides several additional
environments that can be used to typeset complex equations:
\begin{enumerate}
  \item An equation can be spread over multiple lines using the
    \texttt{multline} environment:
    \begin{multline}
      a + b + c + d + e + f + b + c + d + e + f + b + c + d + e +
f \\
      + f + g + h + i + j + k + l + m + n + o + p + q
    \end{multline}

  \item Several aligned equations can be typeset using the
    \texttt{align} environment:
    \begin{align}
              a + b &= c + d     \\
                  u &= v + w + x \\[1ex]
      i + j + k + l &= m
    \end{align}

  \item The \texttt{alignat} environment is similar to
    \texttt{align}, but it doesn't insert horizontal spaces between
    the individual columns:
    \begin{alignat}{2}
      a + b + c &+ d       &   &= 0 \\
              e &+ f + g   &   &= 5
    \end{alignat}

  \item Much like chapter, sections, tables, figures, or list
    items, equations -- such as \eqref{eq:first} and
    \eqref{eq:mine} -- can also be labeled and referenced:
    \begin{alignat}{4}
      b_{11}x_1 &+ b_{12}x_2  &  &+ b_{13}x_3  &  &             &
        &= y_1,                   \label{eq:first} \\
      b_{21}x_1 &+ b_{22}x_2  &  &             &  &+ b_{24}x_4  &
        &= y_2. \tag{My equation} \label{eq:mine}
    \end{alignat}

  \item The \texttt{gather} environment makes it possible to
    typeset several equations without any alignment:
    \begin{gather}
      \psi = \psi\psi, \\
      \eta = \eta\eta\eta\eta\eta\eta, \\
      \theta = \theta.
    \end{gather}

  \item Several cases can be typeset using the \texttt{cases}
    environment:
    \begin{equation}
      |y| = \begin{cases}
        \phantom-y & \text{if }z\geq0, \\
                -y & \text{otherwise}.
      \end{cases}
    \end{equation}
\end{enumerate}
For the complete list of environments and commands, consult the
\textsf{amsmath} package manual\footnote{
  See \url{http://mirrors.ctan.org/macros/latex/required/amslatex/math/amsldoc.pdf}.
  The \texttt{\textbackslash url} command is provided by the
  package \textsf{url}.
}.

\chapter{\textnormal{We \textsf{have} \texttt{several} \textsc{fonts}
  \textit{at} \textbf{disposal}}}
The serified roman font is used for the main body of the text.
\textit{Italics are typically used to denote emphasis or
quotations.} \texttt{The teletype font is typically used for source
code listings.} The \textbf{bold}, \textsc{small-caps} and
\textsf{sans-serif} variants of the base roman font can be used to
denote specific types of information.

\tiny We \scriptsize can \footnotesize also \small change \normalsize
the \large font \Large size, \LARGE although \huge it \Huge
is \huge usually \LARGE not \Large necessary.\normalsize

A wide variety of mathematical fonts is also available, such as: \[
  \mathrm{ABC}, \mathcal{ABC}, \mathbf{ABC}, \mathsf{ABC},
  \mathit{ABC}, \mathtt{ABC}
\] By loading the \textsf{amsfonts} packages, several additional
fonts will become available: \[
  \mathfrak{ABC}, \mathbb{ABC}
\] Many other mathematical fonts are available\footnote{
  See \url{http://tex.stackexchange.com/a/58124/70941}.
}.

\chapter{Inserting the bibliography}
After loading the \texttt{biblatex} package and linking a
bibliography data\-base file to the document using the
\verb"\addbibresource" command, you can start citing the entries.
This is just dummy text \cite{inbook-full} lightly sprinkled with
citations \cite[p.~123]{incollection-full}.  Several sources can be
cited at once \cite{whole-collection, manual-minimal,manual-full}.
\citetitle{inbook-full} was written by \citeauthor{inbook-full} in
\citeyear{inbook-full}. We can also produce \textcite{inbook-full}
or%% Let us define a compound command:
\def\citeauthoryear#1{(\textcite{#1},~\citeyear{#1})}
\citeauthoryear{inbook-full}. The full bibliographic citation is:
\emph{\fullcite{inbook-full}}. We can easily insert a bibliographic
citation into the footnote\footfullcite{inbook-full}.

The \verb"\nocite" command will not generate any
output\nocite{booklet-full}, but it will insert its argument into
the bibliography. The \verb"\nocite{*}" command will insert all the
records in the bibliography database file into the bibliography.
Try uncommenting the command
%% \nocite{*}
and watch the bibliography section come apart at the seams.

When typesetting the document for the first time, citing a
\texttt{work} will expand to [\textbf{work}] and the
\verb"\printbibliography" command will produce no output. It is now
necessary to generate the bibliography by running \texttt{biber
\jobname.bcf} from the command line and then by typesetting the
document again twice. During the first run, the bibliography
section and the citations will be typeset, and in the second run,
the bibliography section will appear in the table of contents.

The \texttt{biber} command needs to be executed from within the
directory, where the \LaTeX\ source file is located. In Windows,
the command line can be opened in a directory by holding down the
\keys{Shift} key and by clicking the right mouse button while
hovering the cursor over a directory.  Select the \menu{Open
Command Window Here} option in the context menu that opens shortly
afterwards.

With online services -- such as Overleaf -- all commands are
executed automatically.

{\csname captions\languagename\endcsname %% Temporarily override
%% the BibLaTeX localization with the original babel definitions.
\makeatletter %% Use the correct localization of the quotations.
  \thesis@selectLocale{\thesis@locale}\makeatother
\printbibliography[heading=bibintoc]} %% Print the bibliography.

\chapter{Inserting the index}
After using the \verb"\makeindex" macro and loading the
\texttt{makeidx} package that provides additional indexing
commands, index entries can be created by issuing the \verb"\index"
command. \index{dummy text|(}It is possible to create ranged index
entries, which will encompass a span of text.\index{dummy text|)}
To insert complex typographic material -- such as $\alpha$
\index{alpha@$\alpha$} or \TeX{} \index{TeX@\TeX} --
into the index, you need to specify a text string, which will
determine how the entry will be sorted. It is also possible to
create hierarchal entries. \index{vehicles!trucks}
\index{vehicles!speed cars}

After typesetting the document, it is necessary to generate the
index by running
\begin{center}%
  \texttt{texindy -I latex -C utf8 -L }$\langle$\textit{locale}%
  $\rangle$\texttt{ \jobname.idx}
\end{center}
from the command line, where $\langle$\textit{locale}$\rangle$
corresponds to the main locale of your thesis -- such as
\texttt{english}, and then typesetting the document again.

The \texttt{texindy} command needs to be executed from within the
directory, where the \LaTeX\ source file is located. In Windows,
the command line can be opened in a directory by holding down the
\keys{Shift} key and by clicking the right mouse button while
hovering the cursor over a directory. Select the \menu{Open Command
Window Here} option in the context menu that opens shortly
afterwards.

With online services -- such as Overleaf -- the commands are
executed automatically, although the locale may be erroneously
detected, or the \texttt{makeindex} tool (which is only able to
sort entries that contain digits and letters of the English
alphabet) may be used instead of \texttt{texindy}. In either case,
the index will be ill-sorted.

\makeatletter\thesis@blocks@clear\makeatother
\phantomsection %% Print the index and insert it into the
\addcontentsline{toc}{chapter}{\indexname} %% table of contents.
\printindex

\appendix %% Start the appendices.
\chapter{An appendix}
Here you can insert the appendices of your thesis.

\end{document}
